\documentclass[unnumsec,webpdf,contemporary,large]{oup-authoring-template}%
%\documentclass[unnumsec,webpdf,contemporary,large,namedate]{oup-authoring-template}% uncomment this line for author year citations and comment the above
%\documentclass[unnumsec,webpdf,contemporary,medium]{oup-authoring-template}
%\documentclass[unnumsec,webpdf,contemporary,small]{oup-authoring-template}

%%%MODERN%%%
%\documentclass[unnumsec,webpdf,modern,large]{oup-authoring-template}
%\documentclass[unnumsec,webpdf,modern,large,namedate]{oup-authoring-template}% uncomment this line for author year citations and comment the above
%\documentclass[unnumsec,webpdf,modern,medium]{oup-authoring-template}
%\documentclass[unnumsec,webpdf,modern,small]{oup-authoring-template}

%%%TRADITIONAL%%%
%\documentclass[unnumsec,webpdf,traditional,large]{oup-authoring-template}
%\documentclass[unnumsec,webpdf,traditional,large,namedate]{oup-authoring-template}% uncomment this line for author year citations and comment the above
%\documentclass[unnumsec,namedate,webpdf,traditional,medium]{oup-authoring-template}
%\documentclass[namedate,webpdf,traditional,small]{oup-authoring-template}

%\onecolumn % for one column layouts

%\usepackage{showframe}

\graphicspath{{Fig/}}

% line numbers
%\usepackage[mathlines, switch]{lineno}
%\usepackage[right]{lineno}

\theoremstyle{thmstyleone}%
\newtheorem{theorem}{Theorem}%  meant for continuous numbers
%%\newtheorem{theorem}{Theorem}[section]% meant for sectionwise numbers
%% optional argument [theorem] produces theorem numbering sequence instead of independent numbers for Proposition
\newtheorem{proposition}[theorem]{Proposition}%
%%\newtheorem{proposition}{Proposition}% to get separate numbers for theorem and proposition etc.
\theoremstyle{thmstyletwo}%
\newtheorem{example}{Example}%
\newtheorem{remark}{Remark}%
\theoremstyle{thmstylethree}%
\newtheorem{definition}{Definition}

\begin{document}

% \journaltitle{Journal Title Here}
\DOI{1234567890}
% \copyrightyear{2022}
% \pubyear{2019}
% \access{Advance Access Publication Date: Day Month Year}
% \appnotes{Paper}

\firstpage{1}

%\subtitle{Subject Section}

\title[]{Final Project}

\author[1]{Nathalie Bonin}
\author[1]{Annie Dai}
\author[1]{Emma Shroyer}

% \authormark{Author Name et al.}

\address[1]{\orgdiv{Computer Science Department}, \orgname{University of Maryland}, \orgaddress{\state{Maryland}, \country{USA}}}

% \corresp[$\ast$]{Corresponding author. \href{email:email-id.com}{email-id.com}}

% \received{Date}{0}{Year}
% \revised{Date}{0}{Year}
% \accepted{Date}{0}{Year}

%\editor{Associate Editor: Name}

%\abstract{
%\textbf{Motivation:} .\\
%\textbf{Results:} .\\
%\textbf{Availability:} .\\
%\textbf{Contact:} \href{name@email.com}{name@email.com}\\
%\textbf{Supplementary information:} Supplementary data are available at \textit{Journal Name}
%online.}

\abstract{Salmon provides a fast and accurate way to estimate transcript abundance. 
This method uses a dual-phase inference algorithm and takes into account fragment biases for transcript verification. 
However, there are some areas where Salmon could improve. \cite{Pascal_et_al._2023}
The inference algorithms it uses, EM and VBEM, are only point estimates and cannot be checked if there is no ground truth available. 
Further, the choice of prior when using the VBEM algorithm can affect the accuracy of the results. In this work, we test how averaging the inference model predictions affect the accuracy of the inference estimation during Salmon’s offline phase.}
\keywords{Salmon, Tuna, Onefish, Twofish, Redfish, Bluefish}
% \boxedtext{
% \begin{itemize}
% \item Key boxed text here.
% \item Key boxed text here.
% \item Key boxed text here.
% \end{itemize}}

\maketitle
\section{Background}
A key characteristic of many newly-formed eukaryotic RNA sequences is the 
inclusion of non-coding and coding regions, which are often referred to as introns and exons. 
Prior to translation, transcript maturation must occur via splicing, 
a process that revolves around the displacement of introns from the mRNA sequence. 
However, depending on environmental conditions and sequence variations, 
some exons are often removed from the mRNA as well, eventually resulting in 
the creation of a different protein isoform. 
Unfortunately, the many factors related to alternative splicing are still poorly understood 
for most eukaryotic proteins, hence the reasoning behind the continuation of transcriptome projects.
\\
An essential facet to transcriptomics is the sequencing techniques used to generate transcript 
fragments for data analysis. Prior approaches to transcriptome quantification involved the use 
of microarrays, which has important limitations such as relatively high risks of cross-hybridization, 
low range of detection, and complicated normalization methods for differential expression analysis 
\cite{Wang_Gerstein_Snyder_2009}. 
RNA-Seq, on the other hand, doesn’t face such limitations and retains microarrays’ 
advantage over classical sequencing such as high-throughput capabilities and the use of inexpensive technology. 
It has been revolutionary in furthering the field of transcriptomics, specifically with regards to eukaryotic cell transcriptomes. 
In particular, recent RNA-seq experiments yielded meaningful results in quantifying expression 
difference across different tissues \cite{Glinos_et_al._2022}, distinguishing host-pathogen interactions 
\cite{Pisu_Huang_Grenier_Russell_2020}, and identifying the effect of structural variants on splicing regulators 
\cite{Pascal_et_al._2023}. 
\\	
Equally important to transcriptomics is abundance quantification. 
Estimations can be done through an alignment to a reference transcriptome sequence; nevertheless, a constant struggle with quantifying eukaryotic samples is to map read fragments back to the original transcript as opposed to the original. Although the detection of protein family abundances can be useful on its own, oftentimes researchers are more so interested in transcript abundances. Since isoforms of the same protein will retain common exons, identifying the source of each read is a tedious process that requires complex statistical methods.
In the past, Salmon was developed to address this particular issue. 
Given a known transcript and a set of sequenced fragments, it quantifies the relative abundance of each transcript in the sample using a dual-phase inference procedure. Salmon improves accuracy compared to other models, because it takes into account sample-specific biases to improve accuracy. When these biases are not accounted for, calculations like the false discovery rate cannot be controlled for. Using multiple inference steps, Salmon improves its abundance estimates using either the VBEM or EM inference algorithms. However, both methods have drawbacks. Both algorithms return point estimates of abundances, and we are uncertain about the returned estimates without ground truth. Further, the choice of  Bayesian priors used in the VBEM algorithm also has considerations. A small prior leads to sparser results than EM. However, a larger prior may result in more estimated non-zero abundance than EM. Prior simulated tests [Salmon documentation] show that VBEM with a small prior can lead to more accurate estimates. However, there has been much research into improving the results of these algorithms such as model averaging. \cite{doi:10.1177/2515245919898657}\cite{11c48783-68bb-36b7-8053-175211b0eaa8}\cite{hoeting_bayesian_1999} Given the trade offs of the two algorithms, we sought to know if the averaged prediction for EM and VBEM inference is better than one algorithm or another. Further, if this ensemble estimate is better, could it be improved by varying the Bayesian priors.
We have found that the average of these outputs [tie in results]

\subsection{Salmon}
Salmon uses a dual-phase inference procedure to provide fast and accurate estimates. 
In an online and offline step, Salmon is able to calculate and improve the abundance 
estimates for a transcript based on fragment GC-content and positional biases. 
Salmon initially utilizes raw reads in a quasi-mapping phase where it performs direct 
quantification instead of enacting a traditional alignment. 
Salmon then moves to an online inference phase. 
Here, Salmon uses a variant of stochastic, collapsed variational Bayesian inference to solve 
a variation Bayesian inference problem. 
In this phase, Salmon estimates the initial expression levels, auxiliary parameters, 
the ‘foreground’ bias models, and fragment equivalence classes.In the offline phase, 
Salmon improves the estimates it calculated in the online phase. Here, the user can 
specify the inference algorithm of EM or VBEM. The chosen algorithm runs to convergence 
on this data, outputting an optimized array of abundances. 
Optionally, using the converged abundances and fragment equivalence classes, 
Salmon can also draw and save estimates from the posterior distribution using Gibbs or bootstrap sampling. \cite{patro_salmon_2017}
\subsection{EM \& VBEM}

The Expectation Maximization (EM) algorithm optimizes the likelihood of the parameters given the data. This algorithm returns a point estimate of the abundances. 
This algorithm is  the default of Salmon.
\\
The Variational Bayesian Expectation Maximization (VBEM) algorithm accounts for the sparsity of the data. It takes a Bayesian nucleotide prior that controls for the sparsity of the data. The default prior for Salmon was $1 \times 10 ^{-2}$.

Each transcriptome \(\mathcal{T}\) is composed of transcripts \(t\).
Each transcript is a nucleotide sequence which can be described through
its length \(l\), effective length \(\tilde l\), and its count
\(c\)---which is the number of times that \(t\) occurs in a given
sample. There are \(M\) transcripts in a given transcriptome.

The probability that a given fragment originates from a transcript \(t\)
depends on the length of that transcript relative to all other
transcripts in the transcriptome. We define this nucleotide fraction
\(\eta\) for a given transcript \(t\) as

\[\eta_i= \dfrac{c_i \cdot \tilde l _i}{\sum^{M}_{\pmb{j} = 1}c_j \cdot \tilde l _j}\]

We obtain a transcript fraction \(\tau\) for a given transcript \(t\) by
normalizing its nucleotide fraction against the effective length of all
transcripts.

\[\eta_i= \dfrac{c_i \cdot \tilde l _i}{\sum^{M}_{\pmb{j} = 1}c_j \cdot \tilde l _j}\]

Let's say that the true nucleotide fraction for a transcript \(t\) is
\(\pmb{\eta}\). We can describe the probability of observing a set of
sequenced fragments \(\mathcal{F}\) as a

\[
$$	
\begin{align*}
	\Pr\left( \mathcal{F}| \pmb{\eta}, Z, \mathcal{T} \right) &= \prod^{N}_{j=1} \Pr\left( f_j| \pmb{\eta}, Z, \mathcal{T} \right)\\
	&= \prod^{N}_{j=1}\sum^{M}_{i=1} \Pr\left( t_i| \pmb{\eta} \cdot \Pr\left( f_j|t_i,z _{ij} =1 \right)  \right)
\end{align*}
$$
\]

where \(N\) is the number of fragments in \(\mathcal{F}\), \(Z\) is a
relationship matrix where \(z _{ij} =1\) when fragment \(f_j\) is
derived from \(t_i\).

We want to obtain \(\pmb{\alpha}\), which is the estimated number of
reads from each transcript. We describe the maximum likelihood estimates
as:

\[\mathcal{L} \left( \pmb{\alpha} |\mathcal{F}, \pmb{Z},\mathcal{T} \right)  = \prod^{N}_{j=1} \sum^{M}_{i=1} \hat \eta_i \Pr\left(  f_j|t_i\right)\]

Written in terms of equivalence classes \(\pmb{C}\)

\[\mathcal{L} \left( \pmb{\alpha} |\mathcal{F}, \pmb{Z},\mathcal{T} \right)  = \prod^{}_{\mathcal{C}^j \in \pmb{C}} \left( \sum_{t_i \in  \pmb{t}^j} \hat \eta_i w_t^j\right) ^{d^{j}}\]

The abundances \(\pmb{\hat \eta}\) are computed directly from
\(\pmb{\alpha}\)

\[\hat \eta_i= \dfrac{\alpha_i}{\sum_j \alpha_j}\]

Then we apply an update function

\[\alpha_i ^{u+1} = \sum^{}_{\mathcal{C}^j \in \pmb{\mathcal{C}}} d ^{j} \left( \dfrac{\alpha_i^u w_i^j}{\sum^{}_{t_k \in \pmb{t}}} j \alpha^u_k w_k^j\right)\]

Until the maximum relative difference in \( \pmb{\alpha}\) is

\[\Delta \left( \alpha^u, \alpha ^{u+1} \right) = \mathrm{max} \dfrac{\left| a_i^u  - \alpha_i ^{u+1}\right| }{ \alpha_i ^{u+1}} < 1 \times 10 ^{-2}\]

for all \(\alpha_i ^{u+1} > 1 \times 10 ^{-8}\), at which point we
derive the estimated nucleotide fraction

\[\hat \eta_i =\dfrac{\alpha_i ^{\prime} }{ \sum_j \alpha ^{\prime} _j}\]

Variational Bayes Optimization

Optionally, the we can apply variational bayeseian optimization where
the update function is

\[\alpha_i ^{u+1} = \sum^{}_{\mathcal{C}^j \in \pmb{\mathcal{C}}} d ^{j} \left( \dfrac{e ^{ \gamma ^u_i} e_i^j}{\sum^{}_{t_k \in \pmb{t}} j e ^{\gamma ^u _k}w^j_k} \right)\]

where

\[\gamma^u_i = \Psi \left( \sum^{M}_{k=1} \alpha^0_k + \alpha^u+_k \right)\]

where \(\Psi\) is the digamma function and the expected value of
nucleotide fractions can be expressed as
\[
$$
\begin{align*}
	\mathbb{E} \left( \eta_i \right) &= \dfrac{\alpha_i^0 + \alpha_i ^{\prime} }{\sum^{}_{j} \alpha^0_j + a ^{\prime} _j}= \dfrac{a_i^0 + \alpha_i ^{\prime} }{\hat \alpha^0 +N}\\
	\hat \alpha^0 &= \sum^{M}_{i = 1} \alpha_i^0
\end{align*}
$$
\]
\subsection{Datasets}
\section{Methods}
\subsection{Combining the traditional EM algorithm with the VBEM algorithm}
Since we are only interested in optimizing abundance estimation in the offline phase, we opted to directly change Salmon v1.10.3’s 
source code. We introduce a new flag, $\mathtt{--useBoth}$, which forces the program to run the offline phase twice: 
once with the traditional EM algorithm and once with the VBEM algorithm. 
The program then computes the average of the two different estimates for the $\pmb{\alpha}$ vector before computing its final 
prediction for $\pmb{\eta}$.

\subsection{Running the new implementation on the polyester dataset}
The commands used for the quantification of each sample using EM, VBEM, or both inference strategies, are provided in Fig \ref{code}.
\begin{figure*}[!t]%
    \centering
    {        
        \begin{verbatim}
        salmon quant -i human_index -l IU -1 sample_1.fa.gz -2 sample_2.fa.gz -p 6 --gcBias --useEM \
            --validateMappings -o sample
        salmon quant -i human_index -l IU -1 sample_1.fa.gz -2 sample_2.fa.gz -p 6 --gcBias --vbPrior \
            1e{from -5 to 1} --validateMappings -o sample
        salmon quant -i human_index -l IU -1 sample_1.fa.gz -2 sample_2.fa.gz -p 6 --gcBias --useBoth \
            --vbPrior 1e{from -5 to 1} --validateMappings -o sample
    \end{verbatim}
    }
    \caption{Commands used for the quantification of each sample using EM, VBEM, or both inference strategies, respectively.}
    \label{code}
\end{figure*}

Quantification was run on the 24 samples through a slurm batch script. 
For each run, 5GB of memory was allocated on the CBCB’s Nexus partition. 
In total, every sample had 15 runs, each with different specifications for the inference method (1 EM, 7 VBEM, and 7 combo) 
and the prior value if applicable ($10^{-5}$ to $10^1$).  All other parameters were kept the same among the 360 total runs: 
\begin{enumerate}
    \item We kept the selective alignment feature on for this experiment. This strategy is currently the default for the pipeline, 
    and it allows Salmon to adopt a more sensitive approach during quasi-mapping.
    \item In recent versions of Salmon, the library type can be automatically detected. Nevertheless, we elected to 
    specify the library type for all the runs.
    \item The $\mathtt{–gcBias}$ flag was passed to the pipeline as well, which allows Salmon to identify and  remedy for 
    GC biases. This extra step is essential for the analysis of our synthetic data since they were simulated from an already 
    existing GC bias profile \cite{love_swimming_2018}.
    
\end{enumerate}

\section{Experiment}

We began by editing Salmon’s source code. We wrote a bash script that downloaded our datasets. We also had a script that ran our samples using EM, VBEM, and a combination of the VBEM and EM algorithms. 
First we ran unaltered Salmon using EM and VBEM. For each pass that used VBEM, we specified a prior 10-5 through 101. 
	We then ran the same datasets using the average of both the EM and VBEM algorithms. We ran the EM and VBEM algorithms on the same data, producing an array. FOr each index in this array, the data was averaged and inserted into a new array that was used the final calculations. We again used priors of 10-5 through 101 when considering the VBEM algorithm. 
	We ran this on [Nathalie’s laptop]
	Finally, we compared the results of these passes using the ground truth of the synthetic data. We calculate the MARD of our results. We also calculated the Spearman Correlation and a two-sided Mann-Whitney U test. 

\section{Results}
We encountered different results depending on the prior selected. Overall, we can separate the outcome of combining both inference algorithms into two categories. The first of such categories includes runs with a prior between $10^{-5}$ and $10^1$, inclusive. As shown in figures \ref{speartable} and \ref{mardtable}, both the $\mathrm{MARD}$ values and the Spearman correlation values are lower for estimates inferred from VBEM than for estimates inferred from both methods (Mann-Whitney U test, $P\leq 1.436 \cdot  10^{-7}$ for MARD, $P < 6.202\cdot 10^{-14}$ for Spearman). However, MARD values tended to be similar between estimates inferred from EM and estimates inferred from both methods (Mann-Whitney U test, $P\geq 0.4803$), whereas the Spearman correlation values were slightly higher for EM (Mann-Whitney U test, $P\leq 0.03941$). Because the MARD metrics are negatively impacted by false discovery rates \cite{patro_salmon_2017}, we hypothesize that the results gathered from running both inference algorithms is affected by EM’s tendency to collect more non-zero abundances than VBEM at low prior values. 

This can also be concluded from the analysis of our second category, which includes runs with a prior of 1 or 10. Unlike with the first category, the $\mathrm{MARD}$ and the Spearman correlation values are strikingly lower for the estimates inferred from EM (Mann-Whitney U test, $P < 6.202\cdot 10^{-14}$ for $\mathrm{MARD}$, $P < 6.202\cdot 10^{-14}$ for Spearman), whereas the difference in MARD between VBEM and the combination of both methods was insignificant for runs with a prior of 1 (Mann-Whitney U test, $P = 0.8944$). The difference between the Spearman correlation values and between the MARD values for runs with a prior of 10 was still significant (Mann-Whitney U test, $P = 5.675 \cdot 10^{-11}$ for $\mathrm{MARD}$, $P < 6.202\cdot 10^{-14}$ for Spearman). We predict that this would suggest that for high prior values, the negative impact from false discovery rates by the VBEM algorithm is offsetted by the more accurate abundance estimates from the EM algorithm. 
\begin{table*}[t]
    \caption{Spearman correlation table for first four samples and their subsamples.
    \label{speartable}}
    \tabcolsep=0pt%%
    \begin{tabular*}{\textwidth}{@{\extracolsep{\fill}}lcccccc@{\extracolsep{\fill}}}
    \toprule
    \begin{tabular}{lrrrrrrrr}
        \textbf{} &
          \multicolumn{1}{l}{Sample $1_{01}$} &
          \multicolumn{1}{l}{Sample $1_{02}$} &
          \multicolumn{1}{l}{Sample $2_{01}$} &
          \multicolumn{1}{l}{Sample $2_{02}$} &
          \multicolumn{1}{l}{Sample $3_{01}$} &
          \multicolumn{1}{l}{Sample $3_{02}$} &
          \multicolumn{1}{l}{Sample $4_{01}$} &
          \multicolumn{1}{l}{Sample $4_{02}$} \\
        EM           & 0.932 & 0.932 & 0.933 & 0.932 & 0.937 & 0.936 & 0.934 & 0.933 \\
        both - $10^{-5}$ & 0.933 & 0.933 & 0.934 & 0.933 & 0.939 & 0.938 & 0.935 & 0.934 \\
        VBEM - $10^{-5}$ & 0.952 & 0.951 & 0.953 & 0.951 & 0.957 & 0.956 & 0.954 & 0.952 \\
        both - $10^{-4}$ & 0.933 & 0.933 & 0.934 & 0.933 & 0.939 & 0.938 & 0.935 & 0.934 \\
        VBEM - $10^{-4}$ & 0.952 & 0.951 & 0.953 & 0.951 & 0.957 & 0.956 & 0.954 & 0.952 \\
        both - $10^{-3}$ & 0.933 & 0.933 & 0.934 & 0.933 & 0.939 & 0.938 & 0.935 & 0.934 \\
        VBEM - $10^{-3}$ & 0.952 & 0.951 & 0.953 & 0.951 & 0.957 & 0.956 & 0.954 & 0.952 \\
        both - $10^{-2}$ & 0.933 & 0.933 & 0.934 & 0.933 & 0.939 & 0.938 & 0.935 & 0.934 \\
        VBEM - $10^{-2}$ & 0.952 & 0.951 & 0.953 & 0.952 & 0.957 & 0.956 & 0.954 & 0.952 \\
        both - $10^{-1}$ & 0.933 & 0.933 & 0.934 & 0.933 & 0.939 & 0.937 & 0.935 & 0.934 \\
        VBEM - $10^{-1}$ & 0.951 & 0.950 & 0.953 & 0.951 & 0.957 & 0.955 & 0.954 & 0.951 \\
        both - $10^{0}$  & 0.755 & 0.754 & 0.755 & 0.754 & 0.755 & 0.755 & 0.755 & 0.754 \\
        VBEM - $10^{0}$  & 0.751 & 0.751 & 0.751 & 0.751 & 0.752 & 0.752 & 0.751 & 0.751 \\
        both - $10^{1}$  & 0.736 & 0.735 & 0.736 & 0.736 & 0.738 & 0.737 & 0.736 & 0.736 \\
        VBEM - $10^{1}$  & 0.715 & 0.715 & 0.715 & 0.716 & 0.717 & 0.717 & 0.715 & 0.715 \\
        \botrule    
    \end{tabular}
    
    \end{tabular*}
    \end{table*}
\section{Statistical Evaluation}
The statistical evaluation of the different inference algorithms was performed with R. 
Because only the original simulated read counts were made available to, we conducted our comparisons on the estimated number of reads 
given by Salmon. We chose two statistical metrics for our analysis:
The mean absolute relative difference, or $\mathrm{MARD}$, 
was already introduced in the original Salmon publication. 
It is the average of the absolute relative difference $\mathrm{ARD}$ for each transcript $i$, 
where $x_i$ is the original simulated read count, and $y_i$ is the read count reported by Salmon.
\[
    \mathrm{ARD}_i = \begin{cases}
        0 &\quad\text{if}\quad x_i = y_i = 0\\
        \dfrac{\left| x_i-y_i \right| }{x_i-y_i} &\quad\text{otherwise}\quad
    \end{cases}
\]
\[
    \mathrm{MARD} = \frac{1}{M} \sum^{M}_{i = 1}\mathrm{ARD}_i
\]
The Spearman correlation, also described in the original publication, calculates the relationship between the rank for our results and the rank for the simulated data. It was computed through the core.test function where we indicated a two sided alternative hypothesis.


\subsection{Spearman Correlation}
\[
    \rho = 1 - \dfrac{6 \sum{d^2_i}}{n(n^2-1)}
\]
\subsection{Mann–Whitney U Test}

MATH
Additionally, we were also required to measure the differences between the three inference algorithms. 
We used the Mann-Whitney test (also known as the Wilcoxon test) 
to calculate the p value for such analysis, and computations were also performed in R 
with the wilcox.test function

\section{Discussion}

We were unable to expand on our project with the time given but in future work, we would like to expand upon ways to best combine the results of the EM and VBEM algorithms. Currently we are only taking the mean of the two results. However, we would also like to experiment with the weighted average of these two algorithms also taking into account various prior sizes. We would like to see if this also improves the estimates of Salmon. 

We would also like to verify the time it takes to run our combined approach. We would like to verify that running this approach will not greatly impact the runtime of Salmon. While we expect the runtime to increase; however, we would like to verify that it is not detrimental to the use of Salmon. 

In future work, it will be beneficial to get a measure of uncertainty when running our trials. We were unable to run this during our initial experiments. However, we would like to know if there is a level of uncertainty with our results when we use the average of the EM and VBEM algorithms. We would like to know if this uncertainty increases or decreases, and if it does so significantly. 


% \section{Introduction}
% The introduction introduces the context and summarizes the manuscript. It is importantly to clearly state the contributions
% of this piece of work. Lorem ipsum dolor sit amet, consectetur adipiscing elit, sed do eiusmod tempor incididunt ut labore et dolore magna aliqua. Ut enim ad minim veniam, quis nostrud exercitation ullamco laboris nisi ut aliquip ex ea commodo consequat. Duis aute irure dolor in reprehenderit in voluptate velit esse cillum dolore eu fugiat nulla pariatur. Excepteur sint occaecat cupidatat non proident, sunt in culpa qui officia deserunt mollit anim id est laborum.

% This is an example of a new parapgraph with a numbered footnote\footnote{\url{https://data.gov.uk/}} and a second footnote marker.\footnote{Example of footnote text.}



% \section{This is an example for first level head - section head}\label{sec2}

% Lorem ipsum dolor sit amet, consectetur adipiscing elit, sed do eiusmod tempor incididunt ut labore et dolore magna aliqua. Ut enim ad minim veniam, quis nostrud exercitation ullamco laboris nisi ut aliquip ex ea commodo consequat. Duis aute irure dolor in reprehenderit in voluptate velit esse cillum dolore eu fugiat nulla pariatur. Excepteur sint occaecat cupidatat non proident, sunt in culpa qui officia deserunt mollit anim id est laborum (refer Section~\ref{sec5}).

% \subsection{This is an example for second level head - subsection head}\label{subsec1}

% Lorem ipsum dolor sit amet, consectetur adipiscing elit, sed do eiusmod tempor incididunt ut labore et dolore magna aliqua. Ut enim ad minim veniam, quis nostrud exercitation ullamco laboris nisi ut aliquip ex ea commodo consequat. Duis aute irure dolor in reprehenderit in voluptate velit esse cillum dolore eu fugiat nulla pariatur. Excepteur sint occaecat cupidatat non proident, sunt in culpa qui officia deserunt mollit anim id est laborum.

% \subsubsection{This is an example for third level head - subsubsection head}\label{subsubsec1}

% Lorem ipsum dolor sit amet, consectetur adipiscing elit, sed do
% eiusmod tempor incididunt ut labore et dolore magna aliqua. Ut enim ad minim veniam, quis nostrud exercitation ullamco laboris nisi ut aliquip ex ea commodo consequat. %Duis aute irure dolor in reprehenderit in voluptate velit esse cillum dolore eu fugiat nulla pariatur. Excepteur sint occaecat cupidatat non proident, sunt in culpa qui officia deserunt mollit anim id est laborum.

% \paragraph{This is an example for fourth level head - paragraph head}

% Lorem ipsum dolor sit amet, consectetur adipiscing elit, sed do eiusmod tempor incididunt ut labore et dolore magna aliqua. Ut enim ad minim veniam, quis nostrud exercitation ullamco laboris nisi ut aliquip ex ea commodo consequat. Duis aute irure dolor in reprehenderit in voluptate velit esse cillum dolore eu fugiat nulla pariatur. Excepteur sint occaecat cupidatat non proident, sunt in culpa qui officia deserunt mollit anim id est laborum.

% \section{This is an example for first level head}\label{sec3}

% \subsection{This is an example for second level head - subsection head}\label{subsec2}

% \subsubsection{This is an example for third level head - subsubsection head}\label{subsubsec2}

% Lorem ipsum dolor sit amet, consectetur adipiscing elit, sed do eiusmod tempor incididunt ut labore et dolore magna aliqua. Ut enim ad minim veniam, quis nostrud exercitation ullamco laboris nisi ut aliquip ex ea commodo consequat. Duis aute irure dolor in reprehenderit in voluptate velit esse cillum dolore eu fugiat nulla pariatur. Excepteur sint occaecat cupidatat non proident, sunt in culpa qui officia deserunt mollit anim id est laborum.

% \paragraph{This is an example for fourth level head - paragraph head}

% Lorem ipsum dolor sit amet, consectetur adipiscing elit, sed do eiusmod tempor incididunt ut labore et dolore magna aliqua. Ut enim ad minim veniam, quis nostrud exercitation ullamco laboris nisi ut aliquip ex ea commodo consequat. Duis aute irure dolor in reprehenderit in voluptate velit esse cillum dolore eu fugiat nulla pariatur. Excepteur sint occaecat cupidatat non proident, sunt in culpa qui officia deserunt mollit anim id est laborum.


% \section{Equations}\label{sec4}

% Equations in \LaTeX{} can either be inline or set as display equations. For
% inline equations use the \verb+$...$+ commands. Eg: the equation
% $H\psi = E \psi$ is written via the command \verb+$H \psi = E \psi$+.

% For display equations (with auto generated equation numbers)
% one can use the equation or eqnarray environments:
% \begin{equation}
% \|\tilde{X}(k)\|^2 \leq\frac{\sum\limits_{i=1}^{p}\left\|\tilde{Y}_i(k)\right\|^2+\sum\limits_{j=1}^{q}\left\|\tilde{Z}_j(k)\right\|^2 }{p+q},\label{eq1}
% \end{equation}
% where,
% \begin{align}
% D_\mu &=  \partial_\mu - ig \frac{\lambda^a}{2} A^a_\mu \nonumber \\
% F^a_{\mu\nu} &= \partial_\mu A^a_\nu - \partial_\nu A^a_\mu + g f^{abc} A^b_\mu A^a_\nu.\label{eq2}
% \end{align}
% Notice the use of \verb+\nonumber+ in the align environment at the end
% of each line, except the last, so as not to produce equation numbers on
% lines where no equation numbers are required. The \verb+\label{}+ command
% should only be used at the last line of an align environment where
% \verb+\nonumber+ is not used.
% \begin{equation}
% Y_\infty = \left( \frac{m}{\textrm{GeV}} \right)^{-3}
%     \left[ 1 + \frac{3 \ln(m/\textrm{GeV})}{15}
%     + \frac{\ln(c_2/5)}{15} \right].
% \end{equation}
% The class file also supports the use of \verb+\mathbb{}+, \verb+\mathscr{}+ and
% \verb+\mathcal{}+ commands. As such \verb+\mathbb{R}+, \verb+\mathscr{R}+
% and \verb+\mathcal{R}+ produces $\mathbb{R}$, $\mathscr{R}$ and $\mathcal{R}$
% respectively (refer Subsubsection~\ref{subsubsec3}).


% Lorem ipsum dolor sit amet, consectetur adipiscing elit, sed do
% eiusmod tempor incididunt ut labore et dolore magna aliqua. Ut enim ad minim veniam, quis nostrud exercitation ullamco laboris nisi ut aliquip ex ea commodo consequat. Duis aute irure dolor in reprehenderit in voluptate velit esse cillum dolore eu fugiat nulla pariatur. Excepteur sint occaecat cupidatat non proident, sunt in culpa qui officia deserunt mollit anim id est laborum. Lorem ipsum dolor sit amet, consectetur adipiscing elit, sed do
% eiusmod tempor incididunt ut labore et dolore magna aliqua. Ut enim ad minim veniam, quis nostrud exercitation ullamco laboris nisi ut aliquip ex ea commodo consequat. Duis aute irure dolor in reprehenderit in voluptate velit esse cillum dolore eu fugiat nulla pariatur. Excepteur sint occaecat cupidatat non proident, sunt in culpa qui officia deserunt mollit anim id est laborum. 
% Lorem ipsum dolor sit amet, consectetur adipiscing elit, sed do
% eiusmod tempor incididunt ut labore et dolore magna aliqua. Ut enim ad minim veniam, quis nostrud exercitation ullamco laboris nisi ut aliquip ex ea commodo consequat. 


% \section{Tables}\label{sec5}

% Tables can be inserted via the normal table and tabular environment. To put
% footnotes inside tables one has to Lorem ipsum dolor sit amet, consectetur adipiscing elit, sed do eiusmod tempor incididunt ut labore et dolore magna aliqua. Ut enim ad minim veniam, quis nostrud exercitation ullamco laboris nisi ut aliquip ex ea commodo consequat. Duis aute irure dolor in reprehenderit in voluptate velit esse cillum dolore eu fugiat nulla pariatur. Excepteur sint occaecat cupidatat non proident, sunt in culpa qui officia deserunt mollit anim id est laborum. use the additional ``tablenotes" environment
% enclosing the tabular environment. The footnote appears just below the table
% itself (refer Tables~\ref{tab1} and \ref{tab2}).


% \begin{verbatim}
% \begin{table}[t]
% \begin{center}
% \begin{minipage}{<width>}
% \caption{<table-caption>\label{<table-label>}}%
% \begin{tabular}{@{}llll@{}}
% \toprule
% column 1 & column 2 & column 3 & column 4\\
% \midrule
% row 1 & data 1 & data 2          & data 3 \\
% row 2 & data 4 & data 5$^{1}$ & data 6 \\
% row 3 & data 7 & data 8      & data 9$^{2}$\\
% \botrule
% \end{tabular}
% \begin{tablenotes}%
% \item Source: Example for source.
% \item[$^{1}$] Example for a 1st table footnote.
% \item[$^{2}$] Example for a 2nd table footnote.
% \end{tablenotes}
% \end{minipage}
% \end{center}
% \end{table}
% \end{verbatim}


% Lengthy tables which do not fit within textwidth should be set as rotated tables. For this, we need to use \verb+\begin{sidewaystable}...+ \verb+\end{sidewaystable}+ instead of\break \verb+\begin{table}...+ \verb+\end{table}+ environment.


% \begin{table}[!t]
% \caption{Caption text\label{tab1}}%
% \begin{tabular*}{\columnwidth}{@{\extracolsep\fill}llll@{\extracolsep\fill}}
% \toprule
% column 1 & column 2  & column 3 & column 4\\
% \midrule
% row 1    & data 1   & data 2  & data 3  \\
% row 2    & data 4   & data 5$^{1}$  & data 6  \\
% row 3    & data 7   & data 8  & data 9$^{2}$  \\
% \botrule
% \end{tabular*}
% \begin{tablenotes}%
% \item Source: This is an example of table footnote this is an example of table footnote this is an example of table footnote this is an example of~table footnote this is an example of table footnote
% \item[$^{1}$] Example for a first table footnote.
% \item[$^{2}$] Example for a second table footnote.
% \end{tablenotes}
% \end{table}

% \begin{table*}[t]
% \caption{Example of a lengthy table which is set to full textwidth.\label{tab2}}
% \tabcolsep=0pt%%
% \begin{tabular*}{\textwidth}{@{\extracolsep{\fill}}lcccccc@{\extracolsep{\fill}}}
% \toprule%
% & \multicolumn{3}{@{}c@{}}{Element 1$^{1}$} & \multicolumn{3}{@{}c@{}}{Element 2$^{2}$} \\
% \cline{2-4}\cline{5-7}%
% Project & Energy & $\sigma_{calc}$ & $\sigma_{expt}$ & Energy & $\sigma_{calc}$ & $\sigma_{expt}$ \\
% \midrule
% Element 3  & 990 A & 1168 & $1547\pm12$ & 780 A & 1166 & $1239\pm100$\\
% Element 4  & 500 A & 961  & $922\pm10$  & 900 A & 1268 & $1092\pm40$\\
% \botrule
% \end{tabular*}
% \begin{tablenotes}%
% \item Note: This is an example of table footnote this is an example of table footnote this is an example of table footnote this is an example of~table footnote this is an example of table footnote
% \item[$^{1}$] Example for a first table footnote.
% \item[$^{2}$] Example for a second table footnote.\vspace*{6pt}
% \end{tablenotes}
% \end{table*}

% \begin{sidewaystable}%[!p]
% \caption{Tables which are too long to fit, should be written using the ``sidewaystable" environment as shown here\label{tab3}}
% \tabcolsep=0pt%
% \begin{tabular*}{\textwidth}{@{\extracolsep{\fill}}lcccccc@{\extracolsep{\fill}}}
% \toprule%
% & \multicolumn{3}{@{}c@{}}{Element 1$^{1}$}& \multicolumn{3}{@{}c@{}}{Element$^{2}$} \\
% \cline{2-4}\cline{5-7}%
% Projectile & Energy     & $\sigma_{calc}$ & $\sigma_{expt}$ & Energy & $\sigma_{calc}$ & $\sigma_{expt}$ \\
% \midrule
% Element 3 & 990 A & 1168 & $1547\pm12$ & 780 A & 1166 & $1239\pm100$ \\
% Element 4 & 500 A & 961  & $922\pm10$  & 900 A & 1268 & $1092\pm40$ \\
% \botrule
% \end{tabular*}
% \begin{tablenotes}%
% \item Note: This is an example of a table footnote this is an example of a table footnote this is an example of a table footnote this is an example of a table footnote this is an example of a table footnote
% \item[$^{1}$] This is an example of a table footnote
% \end{tablenotes}
% \end{sidewaystable}


% \section{Figures}\label{sec6}

% As per display \LaTeX\ standards one has to use eps images for \verb+latex+ compilation and \verb+pdf/jpg/png+ images for
% \verb+pdflatex+ compilation. This is one of the major differences between \verb+latex+
% and \verb+pdflatex+. The images should be single-page documents. The command for inserting images
% for \verb+latex+ and \verb+pdflatex+ can be generalized. The package used to insert images in \verb+latex/pdflatex+ is the
% graphicx package. Figures can be inserted via the normal figure environment as shown in the below example:


% \begin{figure}[!t]%
% \centering
% {\color{black!20}\rule{213pt}{37pt}}
% \caption{This is a widefig. This is an example of a long caption this is an example of a long caption  this is an example of a long caption this is an example of a long caption}\label{fig1}
% \end{figure}

% \begin{figure*}[!t]%
% \centering
% {\color{black!20}\rule{438pt}{74pt}}
% \caption{This is a widefig. This is an example of a long caption this is an example of a long caption  this is an example of a long caption this is an example of a long caption}\label{fig2}
% \end{figure*}


% \begin{verbatim}
% \begin{figure}[t]
%         \centering\includegraphics{<eps-file>}
%         \caption{<figure-caption>}
%         \label{<figure-label>}
% \end{figure}
% \end{verbatim}

% Test text here.

% For sample purposes, we have included the width of images in the
% optional argument of \verb+\includegraphics+ tag. Please ignore this.
% Lengthy figures which do not fit within textwidth should be set in rotated mode. For rotated figures, we need to use \verb+\begin{sidewaysfigure}+ \verb+...+ \verb+\end{sidewaysfigure}+ instead of the \verb+\begin{figure}+ \verb+...+ \verb+\end{figure}+ environment.

% \begin{sidewaysfigure}%
% \centering
% {\color{black!20}\rule{610pt}{102pt}}
% \caption{This is an example for a sideways figure. This is an example of a long caption this is an example of a long caption  this is an example of a long caption this is an example of a long caption}\label{fig3}
% \end{sidewaysfigure}



% \section{Algorithms, Program codes and Listings}\label{sec7}

% Packages \verb+algorithm+, \verb+algorithmicx+ and \verb+algpseudocode+ are used for setting algorithms in latex.
% For this, one has to use the below format:


% \begin{verbatim}
% \begin{algorithm}
% \caption{<alg-caption>}\label{<alg-label>}
% \begin{algorithmic}[1]
% . . .
% \end{algorithmic}
% \end{algorithm}
% \end{verbatim}


% You may need to refer to the above-listed package documentations for more details before setting an \verb+algorithm+ environment.
% To set program codes, one has to use the \verb+program+ package. We need to use the \verb+\begin{program}+ \verb+...+
% \verb+\end{program}+ environment to set program codes.

% \begin{algorithm}[!t]
% \caption{Calculate $y = x^n$}\label{algo1}
% \begin{algorithmic}[1]
% \Require $n \geq 0 \vee x \neq 0$
% \Ensure $y = x^n$
% \State $y \Leftarrow 1$
% \If{$n < 0$}
%         \State $X \Leftarrow 1 / x$
%         \State $N \Leftarrow -n$
% \Else
%         \State $X \Leftarrow x$
%         \State $N \Leftarrow n$
% \EndIf
% \While{$N \neq 0$}
%         \If{$N$ is even}
%             \State $X \Leftarrow X \times X$
%             \State $N \Leftarrow N / 2$
%         \Else[$N$ is odd]
%             \State $y \Leftarrow y \times X$
%             \State $N \Leftarrow N - 1$
%         \EndIf
% \EndWhile
% \end{algorithmic}
% \end{algorithm}

% Similarly, for \verb+listings+, one has to use the \verb+listings+ package. The \verb+\begin{lstlisting}+ \verb+...+ \verb+\end{lstlisting}+ environment is used to set environments similar to the \verb+verbatim+ environment. Refer to the \verb+lstlisting+ package documentation for more details on this.


% \begin{minipage}{\hsize}%
% \lstset{language=Pascal}% Set your language (you can change the language for each code-block optionally)
% \begin{lstlisting}[frame=single,framexleftmargin=-1pt,framexrightmargin=-17pt,framesep=12pt,linewidth=0.98\textwidth]
% for i:=maxint to 0 do
% begin
% { do nothing }
% end;
% Write('Case insensitive ');
% Write('Pascal keywords.');
% \end{lstlisting}
% \end{minipage}


% \section{Cross referencing}\label{sec8}

% Environments such as figure, table, equation, and align can have a label
% declared via the \verb+\label{#label}+ command. For figures and table
% environments one should use the \verb+\label{}+ command inside or just
% below the \verb+\caption{}+ command.  One can then use the
% \verb+\ref{#label}+ command to cross-reference them. As an example, consider
% the label declared for Figure \ref{fig1} which is
% \verb+\label{fig1}+. To cross-reference it, use the command
% \verb+ Figure \ref{fig1}+, for which it comes up as
% ``Figure~\ref{fig1}".

% \subsection{Details on reference citations}\label{subsec3}

% With standard numerical .bst files, only numerical citations are possible.
% With an author-year .bst file, both numerical and author-year citations are possible.

% If author-year citations are selected, \verb+\bibitem+ must have one of the following forms:


% {\footnotesize%
% \begin{verbatim}
% \bibitem[Jones et al.(1990)]{key}...
% \bibitem[Jones et al.(1990)Jones,
%                 Baker, and Williams]{key}...
% \bibitem[Jones et al., 1990]{key}...
% \bibitem[\protect\citeauthoryear{Jones,
%                 Baker, and Williams}
%                 {Jones et al.}{1990}]{key}...
% \bibitem[\protect\citeauthoryear{Jones et al.}
%                 {1990}]{key}...
% \bibitem[\protect\astroncite{Jones et al.}
%                 {1990}]{key}...
% \bibitem[\protect\citename{Jones et al., }
%                 1990]{key}...
% \harvarditem[Jones et al.]{Jones, Baker, and
%                 Williams}{1990}{key}...
% \end{verbatim}}


% This is either to be made up manually, or to be generated by an
% appropriate .bst file with BibTeX. Then,


% {%
% \begin{verbatim}
%                     Author-year mode
%                         || Numerical mode
% \citet{key} ==>>  Jones et al. (1990)
%                         || Jones et al. [21]
% \citep{key} ==>> (Jones et al., 1990) || [21]
% \end{verbatim}}


% \noindent
% Multiple citations as normal:


% {%
% \begin{verbatim}
% \citep{key1,key2} ==> (Jones et al., 1990;
%                          Smith, 1989)||[21,24]
%         or (Jones et al., 1990, 1991)||[21,24]
%         or (Jones et al., 1990a,b)   ||[21,24]
% \end{verbatim}}


% \noindent
% \verb+\cite{key}+ is the equivalent of \verb+\citet{key}+ in author-year mode
% and  of \verb+\citep{key}+ in numerical mode. Full author lists may be forced with
% \verb+\citet*+ or \verb+\citep*+, e.g.


% {%
% \begin{verbatim}
% \citep*{key} ==>> (Jones, Baker, and Mark, 1990)
% \end{verbatim}}


% \noindent
% Optional notes as:


% {%
% \begin{verbatim}
% \citep[chap. 2]{key}     ==>>
%         (Jones et al., 1990, chap. 2)
% \citep[e.g.,][]{key}     ==>>
%         (e.g., Jones et al., 1990)
% \citep[see][pg. 34]{key} ==>>
%         (see Jones et al., 1990, pg. 34)
% \end{verbatim}}


% \noindent
% (Note: in standard LaTeX, only one note is allowed, after the ref.
% Here, one note is like the standard, two make pre- and post-notes.)


% {%
% \begin{verbatim}
% \citealt{key}   ==>> Jones et al. 1990
% \citealt*{key}  ==>> Jones, Baker, and
%                         Williams 1990
% \citealp{key}   ==>> Jones et al., 1990
% \citealp*{key}  ==>> Jones, Baker, and
%                         Williams, 1990
% \end{verbatim}}


% \noindent
% Additional citation possibilities (both author-year and numerical modes):


% {%
% \begin{verbatim}
% \citeauthor{key}       ==>> Jones et al.
% \citeauthor*{key}      ==>> Jones, Baker, and
%                                 Williams
% \citeyear{key}         ==>> 1990
% \citeyearpar{key}      ==>> (1990)
% \citetext{priv. comm.} ==>> (priv. comm.)
% \citenum{key}          ==>> 11 [non-superscripted]
% \end{verbatim}}


% \noindent
% Note: full author lists depend on whether the bib style supports them;
% if not, the abbreviated list is printed even when full is requested.

% \noindent
% For names like della Robbia at the start of a sentence, use


% {%
% \begin{verbatim}
% \Citet{dRob98}      ==>> Della Robbia (1998)
% \Citep{dRob98}      ==>> (Della Robbia, 1998)
% \Citeauthor{dRob98} ==>> Della Robbia
% \end{verbatim}}


% \noindent
% The following is an example for \verb+\cite{...}+: \cite{rahman2019centroidb}. Another example for \verb+\citep{...}+: \citep{bahdanau2014neural,imboden2018cardiorespiratory,motiian2017unified,murphy2012machine,ji20123d}.
% Sample cites here \cite{krizhevsky2012imagenet,horvath2018dna} and \cite{pyrkov2018quantitative}, \cite{wang2018face}, \cite{lecun2015deep,zhang2018fine,ravi2016deep}.


% \section{Lists}\label{sec9}

% List in \LaTeX{} can be of three types: numbered, bulleted and unnumbered. The ``enumerate'' environment produces a numbered list, the 
% ``itemize'' environment produces a bulleted list and the ``unlist''
% environment produces an unnumbered list.
% In each environment, a new entry is added via the \verb+\item+ command.
% \begin{enumerate}[1.]
% \item This is the 1st item

% \item Enumerate creates numbered lists, itemize creates bulleted lists and
% unnumerate creates unnumbered lists.
% \begin{enumerate}[(a)]
% \item Second level numbered list. Enumerate creates numbered lists, itemize creates bulleted lists and
% description creates unnumbered lists.

% \item Second level numbered list. Enumerate creates numbered lists, itemize creates bulleted lists and
% description creates unnumbered lists.
% \begin{enumerate}[(ii)]
% \item Third level numbered list. Enumerate creates numbered lists, itemize creates bulleted lists and
% description creates unnumbered lists.

% \item Third level numbered list. Enumerate creates numbered lists, itemize creates bulleted lists and
% description creates unnumbered lists.
% \end{enumerate}

% \item Second level numbered list. Enumerate creates numbered lists, itemize creates bulleted lists and
% description creates unnumbered lists.

% \item Second level numbered list. Enumerate creates numbered lists, itemize creates bulleted lists and
% description creates unnumbered lists.
% \end{enumerate}

% \item Enumerate creates numbered lists, itemize creates bulleted lists and
% description creates unnumbered lists.

% \item Numbered lists continue.
% \end{enumerate}
% Lists in \LaTeX{} can be of three types: enumerate, itemize and description.
% In each environment, a new entry is added via the \verb+\item+ command.
% \begin{itemize}
% \item First level bulleted list. This is the 1st item

% \item First level bulleted list. Itemize creates bulleted lists and description creates unnumbered lists.
% \begin{itemize}
% \item Second level dashed list. Itemize creates bulleted lists and description creates unnumbered lists.

% \item Second level dashed list. Itemize creates bulleted lists and description creates unnumbered lists.

% \item Second level dashed list. Itemize creates bulleted lists and description creates unnumbered lists.
% \end{itemize}

% \item First level bulleted list. Itemize creates bulleted lists and description creates unnumbered lists.

% \item First level bulleted list. Bullet lists continue.
% \end{itemize}

% \noindent
% Example for unnumbered list items:

% \begin{unlist}
% \item Sample unnumberd list text. Sample unnumberd list text. Sample unnumberd list text. Sample unnumberd list text. Sample unnumberd list text.

% \item Sample unnumberd list text. Sample unnumberd list text. Sample unnumberd list text.

% \item sample unnumberd list text. Sample unnumberd list text. Sample unnumberd list text. Sample unnumberd list text. Sample unnumberd list text. Sample unnumberd list text. Sample unnumberd list text.
% \end{unlist}

% \section{Examples for theorem-like environments}\label{sec10}

% For theorem-like environments, we require the \verb+amsthm+ package. There are three types of predefined theorem styles - \verb+thmstyleone+, \verb+thmstyletwo+ and \verb+thmstylethree+   (check your journal's instructions page in case a specific style is required).

% \medskip
% \noindent\begin{tabular}{|l|p{13pc}|}
% \hline
% \verb+thmstyleone+ & Numbered, theorem head in bold font and theorem text in italic style \\\hline
% \verb+thmstyletwo+ & Numbered, theorem head in roman font and theorem text in italic style \\\hline
% \verb+thmstylethree+ & Numbered, theorem head in bold font and theorem text in roman style \\\hline
% \end{tabular}


% \begin{theorem}[Theorem subhead]\label{thm1}
% Example theorem text. Example theorem text. Example theorem text. Example theorem text. Example theorem text.
% Example theorem text. Example theorem text. Example theorem text. Example theorem text. Example theorem text.
% Example theorem text.
% \end{theorem}

% Quisque ullamcorper placerat ipsum. Cras nibh. Morbi vel justo vitae lacus tincidunt ultrices. Lorem ipsum dolor sit
% amet, consectetuer adipiscing elit. In hac habitasse platea dictumst. Integer tempus convallis augue.

% \begin{proposition}
% Example proposition text. Example proposition text. Example proposition text. Example proposition text. Example proposition text.
% Example proposition text. Example proposition text. Example proposition text. Example proposition text. Example proposition text.
% \end{proposition}

% Nulla malesuada porttitor diam. Donec felis erat, congue non, volutpat at, tincidunt tristique, libero. Vivamus
% viverra fermentum felis. Donec nonummy pellentesque ante.

% \begin{example}
% Phasellus adipiscing semper elit. Proin fermentum massa
% ac quam. Sed diam turpis, molestie vitae, placerat a, molestie nec, leo. Maecenas lacinia. Nam ipsum ligula, eleifend
% at, accumsan nec, suscipit a, ipsum. Morbi blandit ligula feugiat magna. Nunc eleifend consequat lorem.
% \end{example}

% Nulla malesuada porttitor diam. Donec felis erat, congue non, volutpat at, tincidunt tristique, libero. Vivamus
% viverra fermentum felis. Donec nonummy pellentesque ante.

% \begin{remark}
% Phasellus adipiscing semper elit. Proin fermentum massa
% ac quam. Sed diam turpis, molestie vitae, placerat a, molestie nec, leo. Maecenas lacinia. Nam ipsum ligula, eleifend
% at, accumsan nec, suscipit a, ipsum. Morbi blandit ligula feugiat magna. Nunc eleifend consequat lorem.
% \end{remark}

% Quisque ullamcorper placerat ipsum. Cras nibh. Morbi vel justo vitae lacus tincidunt ultrices. Lorem ipsum dolor sit
% amet, consectetuer adipiscing elit. In hac habitasse platea dictumst.

% \begin{definition}[Definition sub head]
% Example definition text. Example definition text. Example definition text. Example definition text. Example definition text. Example definition text. Example definition text. Example definition text.
% \end{definition}

% Apart from the above styles, we have the \verb+\begin{proof}+ \verb+...+ \verb+\end{proof}+ environment - with the proof head in italic style and the body text in roman font with an open square at the end of each proof environment.

% \begin{proof}Example for proof text. Example for proof text. Example for proof text. Example for proof text. Example for proof text. Example for proof text. Example for proof text. Example for proof text. Example for proof text. Example for proof text.
% \end{proof}

% Nam dui ligula, fringilla a, euismod sodales, sollicitudin vel, wisi. Morbi auctor lorem non justo. Nam lacus libero,
% pretium at, lobortis vitae, ultricies et, tellus. Donec aliquet, tortor sed accumsan bibendum, erat ligula aliquet magna,
% vitae ornare odio metus a mi.

% \begin{proof}[Proof of Theorem~{\upshape\ref{thm1}}]
% Example for proof text. Example for proof text. Example for proof text. Example for proof text. Example for proof text. Example for proof text. Example for proof text. Example for proof text. Example for proof text. Example for proof text.
% \end{proof}

% \noindent
% For a quote environment, one has to use\newline \verb+\begin{quote}...\end{quote}+
% \begin{quote}
% Quoted text example. Aliquam porttitor quam a lacus. Praesent vel arcu ut tortor cursus volutpat. In vitae pede quis diam bibendum placerat. Fusce elementum
% convallis neque. Sed dolor orci, scelerisque ac, dapibus nec, ultricies ut, mi. Duis nec dui quis leo sagittis commodo.
% \end{quote}
% Donec congue. Maecenas urna mi, suscipit in, placerat ut, vestibulum ut, massa. Fusce ultrices nulla et nisl (refer Figure~\ref{fig3}). Pellentesque habitant morbi tristique senectus et netus et malesuada fames ac turpis egestas. Etiam ligula arcu,
% elementum a, venenatis quis, sollicitudin sed, metus. Donec nunc pede, tincidunt in, venenatis vitae, faucibus vel (refer Table~\ref{tab3}).

% \section{Conclusion}

% Some Conclusions here.

% %%%%%%%%%%%%%%

% \begin{appendices}

% \section{Section title of first appendix}\label{sec11}

% Nam dui ligula, fringilla a, euismod sodales, sollicitudin vel, wisi. Morbi auctor lorem non justo. Nam lacus libero,
% pretium at, lobortis vitae, ultricies et, tellus. Donec aliquet, tortor sed accumsan bibendum, erat ligula aliquet magna,
% vitae ornare odio metus a mi. Morbi ac orci et nisl hendrerit mollis. Suspendisse ut massa. Cras nec ante. Pellentesque
% a nulla. Cum sociis natoque penatibus et magnis dis parturient montes, nascetur ridiculus mus. Aliquam tincidunt
% urna. Nulla ullamcorper vestibulum turpis. Pellentesque cursus luctus mauris.

% \subsection{Subsection title of first appendix}\label{subsec4}

% Nam dui ligula, fringilla a, euismod sodales, sollicitudin vel, wisi. Morbi auctor lorem non justo. Nam lacus libero,
% pretium at, lobortis vitae, ultricies et, tellus. Donec aliquet, tortor sed accumsan bibendum, erat ligula aliquet magna,
% vitae ornare odio metus a mi. Morbi ac orci et nisl hendrerit mollis. Suspendisse ut massa. Cras nec ante. Pellentesque
% a nulla. Cum sociis natoque penatibus et magnis dis parturient montes, nascetur ridiculus mus. Aliquam tincidunt
% urna. Nulla ullamcorper vestibulum turpis. Pellentesque cursus luctus mauris.

% \subsubsection{Subsubsection title of first appendix}\label{subsubsec3}

% Example for an unnumbered figure:

% \begin{figure}[!h]
% \centering
% {\color{black!20}\rule{85pt}{92pt}}
% \end{figure}

% Fusce mauris. Vestibulum luctus nibh at lectus. Sed bibendum, nulla a faucibus semper, leo velit ultricies tellus, ac
% venenatis arcu wisi vel nisl. Vestibulum diam. Aliquam pellentesque, augue quis sagittis posuere, turpis lacus congue
% quam, in hendrerit risus eros eget felis.

% \section{Section title of second appendix}\label{sec12}%

% Fusce mauris. Vestibulum luctus nibh at lectus. Sed bibendum, nulla a faucibus semper, leo velit ultricies tellus, ac
% venenatis arcu wisi vel nisl. Vestibulum diam. Aliquam pellentesque, augue quis sagittis posuere, turpis lacus congue
% quam, in hendrerit risus eros eget felis. Maecenas eget erat in sapien mattis porttitor. Vestibulum porttitor. Nulla
% facilisi. Sed a turpis eu lacus commodo facilisis. Morbi fringilla, wisi in dignissim interdum, justo lectus sagittis dui, et
% vehicula libero dui cursus dui. Mauris tempor ligula sed lacus. Duis cursus enim ut augue. Cras ac magna. Cras nulla.

% \begin{figure}[b]
% \centering
% {\color{black!20}\rule{217pt}{120pt}}
% \caption{This is an example for appendix figure\label{fig4}}
% \end{figure}

% \begin{table}[t]%
% \begin{center}
% \begin{minipage}{.52\columnwidth}
% \caption{This is an example of Appendix table showing food requirements of army, navy and airforce\label{tab4}}%
% \begin{tabular}{@{}lcc@{}}%
% \toprule
% col1 head & col2 head & col3 head \\
% \midrule
% col1 text & col2 text & col3 text \\
% col1 text & col2 text & col3 text \\
% col1 text & col2 text & col3 text \\
% \botrule
% \end{tabular}
% \end{minipage}
% \end{center}
% \end{table}

% \subsection{Subsection title of second appendix}\label{subsec5}

% Sed commodo posuere pede. Mauris ut est. Ut quis purus. Sed ac odio. Sed vehicula hendrerit sem. Duis non odio.
% Morbi ut dui. Sed accumsan risus eget odio. In hac habitasse platea dictumst. Pellentesque non elit. Fusce sed justo
% eu urna porta tincidunt. Mauris felis odio, sollicitudin sed, volutpat a, ornare ac, erat. Morbi quis dolor. Donec
% pellentesque, erat ac sagittis semper, nunc dui lobortis purus, quis congue purus metus ultricies tellus. Proin et quam.
% Class aptent taciti sociosqu ad litora torquent per conubia nostra, per inceptos hymenaeos. Praesent sapien turpis,
% fermentum vel, eleifend faucibus, vehicula eu, lacus.

% Sed commodo posuere pede. Mauris ut est. Ut quis purus. Sed ac odio. Sed vehicula hendrerit sem. Duis non odio.
% Morbi ut dui. Sed accumsan risus eget odio. In hac habitasse platea dictumst. Pellentesque non elit. Fusce sed justo
% eu urna porta tincidunt. Mauris felis odio, sollicitudin sed, volutpat a, ornare ac, erat. Morbi quis dolor. Donec
% pellentesque, erat ac sagittis semper, nunc dui lobortis purus, quis congue purus metus ultricies tellus. Proin et quam.
% Class aptent taciti sociosqu ad litora torquent per conubia nostra, per inceptos hymenaeos. Praesent sapien turpis,
% fermentum vel, eleifend faucibus, vehicula eu, lacus.

% \subsubsection{Subsubsection title of second appendix}\label{subsubsec4}

% Lorem ipsum dolor sit amet, consectetuer adipiscing elit. Ut purus elit, vestibulum ut, placerat ac, adipiscing vitae,
% felis. Curabitur dictum gravida mauris. Nam arcu libero, nonummy eget, consectetuer id, vulputate a, magna. Donec
% vehicula augue eu neque.


% Example for an equation inside the appendix:
% \begin{align}
%  p &= \frac{\gamma^{2} - (n_{C} -1)H}{(n_{C} - 1) + H - 2\gamma}, \label{1eq:hybobo:pfromgH} \\
%  \theta &= \frac{(\gamma - H)^{2}(\gamma - n_{C} -1)^{2}}{(n_{C} - 1 + H - 2\gamma)^{2}} \label{2eq:hybobo:tfromgH}\; .
% \end{align}

% \section{Example of another appendix section}\label{sec13}%

% Nam dui ligula, fringilla a, euismod sodales, sollicitudin vel, wisi. Morbi auctor lorem non justo. Nam lacus libero,
% pretium at, lobortis vitae, ultricies et, tellus. Donec aliquet, tortor sed accumsan bibendum, erat ligula aliquet magna,
% vitae ornare odio metus a mi. Morbi ac orci et nisl hendrerit mollis. Suspendisse ut massa. Cras nec ante. Pellentesque
% a nulla. Cum sociis natoque penatibus et magnis dis parturient montes, nascetur ridiculus mus. Aliquam tincidunt
% urna. Nulla ullamcorper vestibulum turpis. Pellentesque cursus luctus mauris
% \begin{equation}
% \mathcal{L} = i \bar{\psi} \gamma^\mu D_\mu \psi
%     - \frac{1}{4} F_{\mu\nu}^a F^{a\mu\nu} - m \bar{\psi} \psi.
% \label{eq26}
% \end{equation}

% Nulla malesuada porttitor diam. Donec felis erat, congue non, volutpat at, tincidunt tristique, libero. Vivamus
% viverra fermentum felis. Donec nonummy pellentesque ante. Phasellus adipiscing semper elit. Proin fermentum massa
% ac quam. Sed diam turpis, molestie vitae, placerat a, molestie nec, leo. Maecenas lacinia. Nam ipsum ligula, eleifend
% at, accumsan nec, suscipit a, ipsum. Morbi blandit ligula feugiat magna. Nunc eleifend consequat lorem. Sed lacinia
% nulla vitae enim. Pellentesque tincidunt purus vel magna. Integer non enim. Praesent euismod nunc eu purus. Donec
% bibendum quam in tellus. Nullam cursus pulvinar lectus. Donec et mi. Nam vulputate metus eu enim. Vestibulum
% pellentesque felis eu massa.

% Nulla malesuada porttitor diam. Donec felis erat, congue non, volutpat at, tincidunt tristique, libero. Vivamus
% viverra fermentum felis. Donec nonummy pellentesque ante. Phasellus adipiscing semper elit. Proin fermentum massa
% ac quam. Sed diam turpis, molestie vitae, placerat a, molestie nec, leo. Maecenas lacinia. Nam ipsum ligula, eleifend
% at, accumsan nec, suscipit a, ipsum. Morbi blandit ligula feugiat magna. Nunc eleifend consequat lorem. Sed lacinia
% nulla vitae enim. Pellentesque tincidunt purus vel magna. Integer non enim. Praesent euismod nunc eu purus. Donec
% bibendum quam in tellus. Nullam cursus pulvinar lectus. Donec et mi. Nam vulputate metus eu enim. Vestibulum
% pellentesque felis eu massa.

% Nulla malesuada porttitor diam. Donec felis erat, congue non, volutpat at, tincidunt tristique, libero. Vivamus
% viverra fermentum felis. Donec nonummy pellentesque ante. Phasellus adipiscing semper elit. Proin fermentum massa
% ac quam. Sed diam turpis, molestie vitae, placerat a, molestie nec, leo. Maecenas lacinia. Nam ipsum ligula, eleifend
% at, accumsan nec, suscipit a, ipsum. Morbi blandit ligula feugiat magna. Nunc eleifend consequat lorem. Sed lacinia
% nulla vitae enim. Pellentesque tincidunt purus vel magna. Integer non enim. Praesent euismod nunc eu purus. Donec
% bibendum quam in tellus. Nullam cursus pulvinar lectus. Donec et mi. Nam vulputate metus eu enim. Vestibulum
% pellentesque felis eu massa.

% Nulla malesuada porttitor diam. Donec felis erat, congue non, volutpat at, tincidunt tristique, libero. Vivamus
% viverra fermentum felis. Donec nonummy pellentesque ante. Phasellus adipiscing semper elit. Proin fermentum massa
% ac quam. Sed diam turpis, molestie vitae, placerat a, molestie nec, leo. Maecenas lacinia. Nam ipsum ligula, eleifend
% at, accumsan nec, suscipit a, ipsum. Morbi blandit ligula feugiat magna. Nunc eleifend consequat lorem. Sed lacinia
% nulla vitae enim. Pellentesque tincidunt purus vel magna. Integer non enim. Praesent euismod nunc eu purus. Donec
% bibendum quam in tellus. Nullam cursus pulvinar lectus. Donec et mi. Nam vulputate metus eu enim. Vestibulum
% pellentesque felis eu massa. Donec
% bibendum quam in tellus. Nullam cursus pulvinar lectus. Donec et mi. Nam vulputate metus eu enim. Vestibulum
% pellentesque felis eu massa.

% %% Example for unnumbered table inside appendix
% \begin{table}
% \begin{center}
% \begin{minipage}{.52\columnwidth}
% \caption{}{%
% \begin{tabular}{lcc}%
% \toprule
% col1 head & col2 head & col3 head \\
% \midrule
% col1 text & col2 text & col3 text \\
% col1 text & col2 text & col3 text \\
% col1 text & col2 text & col3 text \\
% \botrule
% \end{tabular}}{}
% \end{minipage}
% \end{center}
% \end{table}

% \end{appendices}

% \section{Competing interests}
% No competing interest is declared.

% \section{Author contributions statement}

% Must include all authors, identified by initials, for example:
% S.R. and D.A. conceived the experiment(s),  S.R. conducted the experiment(s), S.R. and D.A. analysed the results.  S.R. and D.A. wrote and reviewed the manuscript.

\section{Acknowledgments}
The authors thank the anonymous reviewers for their valuable suggestions. This work is supported in part by funds from the National Science Foundation (NSF: \# 1636933 and \# 1920920).
\cite{horvath2018dna}

%\bibliographystyle{plain}
% \bibliography{reference}

\begin{thebibliography}{10}

\bibitem{bahdanau2014neural}
Dzmitry Bahdanau, Kyunghyun Cho, and Yoshua Bengio.
\newblock Neural machine translation by jointly learning to align and
  translate.
\newblock {\em arXiv preprint arXiv:1409.0473}, 2014.

\bibitem{horvath2018dna}
Steve Horvath and Kenneth Raj.
\newblock Dna methylation-based biomarkers and the epigenetic clock theory of
  ageing.
\newblock {\em Nature Reviews Genetics}, 19(6):371, 2018.

\bibitem{imboden2018cardiorespiratory}
Mary~T Imboden, Matthew~P Harber, Mitchell~H Whaley, W~Holmes Finch, Derron~L
  Bishop, and Leonard~A Kaminsky.
\newblock Cardiorespiratory fitness and mortality in healthy men and women.
\newblock {\em Journal of the American College of Cardiology},
  72(19):2283--2292, 2018.

\bibitem{ji20123d}
Shuiwang Ji, Wei Xu, Ming Yang, and Kai Yu.
\newblock 3d convolutional neural networks for human action recognition.
\newblock {\em IEEE Transactions on Pattern Analysis and Machine Intelligence},
  35(1):221--231, 2012.

\bibitem{krizhevsky2012imagenet}
Alex Krizhevsky, Ilya Sutskever, and Geoffrey~E Hinton.
\newblock Imagenet classification with deep convolutional neural networks.
\newblock In {\em Advances in Neural Information Processing Systems}, pages
  1097--1105, 2012.

\bibitem{lecun2015deep}
Yann LeCun, Yoshua Bengio, and Geoffrey Hinton.
\newblock Deep learning.
\newblock {\em Nature}, 521(7553):436, 2015.

\bibitem{motiian2017unified}
Saeid Motiian, Marco Piccirilli, Donald~A Adjeroh, and Gianfranco Doretto.
\newblock Unified deep supervised domain adaptation and generalization.
\newblock In {\em Proceedings of the IEEE International Conference on Computer
  Vision}, pages 5715--5725, 2017.

\bibitem{murphy2012machine}
Kevin~P Murphy.
\newblock {\em Machine learning: A probabilistic perspective}.
\newblock MIT press, 2012.

\bibitem{american2013acsm}
American~College of~Sports~Medicine et~al.
\newblock {\em ACSM's guidelines for exercise testing and prescription}.
\newblock Lippincott Williams \& Wilkins, 2013.

\bibitem{pyrkov2018quantitative}
Timothy~V Pyrkov, Evgeny Getmantsev, Boris Zhurov, Konstantin Avchaciov,
  Mikhail Pyatnitskiy, Leonid Menshikov, Kristina Khodova, Andrei~V Gudkov, and
  Peter~O Fedichev.
\newblock Quantitative characterization of biological age and frailty based on
  locomotor activity records.
\newblock {\em Aging (Albany NY)}, 10(10):2973, 2018.

\bibitem{rahman2019centroidb}
Syed~Ashiqur Rahman and Donald Adjeroh.
\newblock Centroid of age neighborhoods: A generalized approach to estimate
  biological age.
\newblock In {\em 2019 IEEE EMBS International Conference on Biomedical \&
  Health Informatics (BHI)}, pages 1--4. IEEE, 2019.

\bibitem{ravi2016deep}
Daniele Rav{\`\i}, Charence Wong, Fani Deligianni, Melissa Berthelot, Javier
  Andreu-Perez, Benny Lo, and Guang-Zhong Yang.
\newblock Deep learning for health informatics.
\newblock {\em IEEE {J}ournal of {B}iomedical and {H}ealth {I}nformatics},
  21(1):4--21, 2016.

\bibitem{wang2018face}
Zongwei Wang, Xu~Tang, Weixin Luo, and Shenghua Gao.
\newblock Face aging with identity-preserved conditional generative adversarial
  networks.
\newblock In {\em Proceedings of the IEEE Conference on Computer Vision and
  Pattern Recognition}, pages 7939--7947, 2018.

\bibitem{zhang2018fine}
Ke~Zhang, Na~Liu, Xingfang Yuan, Xinyao Guo, Ce~Gao, and Zhenbing Zhao.
\newblock Fine-grained age estimation in the wild with attention {LSTM}
  networks.
\newblock {\em arXiv preprint arXiv:1805.10445}, 2018.

\end{thebibliography}


% %USE THE BELOW OPTIONS IN CASE YOU NEED AUTHOR YEAR FORMAT.
% \bibliographystyle{abbrvnat}
% \bibliography{reference}


\end{document}
