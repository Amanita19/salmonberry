\section{Introduction}

A key characteristic of many newly-formed eukaryotic RNA sequences is the inclusion of non-coding and coding regions, 
which are often referred to as introns and exons. Prior to translation, transcript maturation must occur via splicing, 
a process that revolves around the displacement of introns from the mRNA sequence. However, depending on environmental 
conditions and sequence variations, some exons are often removed from the mRNA as well, eventually resulting in the creation of 
a different protein isoform. Unfortunately, the many factors related to alternative splicing are still poorly understood for most 
eukaryotic proteins, hence the reasoning behind the continuation of transcriptome projects.\\

One essential facet to transcriptomics is the sequencing techniques used to generate transcript fragments for data analysis. 
Prior approaches to transcriptome quantification involved the use of microarrays, which has important limitations such as relatively high 
risks of cross-hybridization, low range of detection, and complicated normalization methods for differential expression analysis 
\cite{Wang_Gerstein_Snyder_2009}. 
RNA-Seq, on the other hand, does not face such limitations and retains microarrays’ advantage over classical sequencing such as 
high-throughput capabilities and the use of inexpensive technology. It has been revolutionary in furthering the field of transcriptomics,
specifically with regards to eukaryotic cell transcriptomes. In particular, recent RNA-seq experiments yielded meaningful results in 
quantifying expression difference across different tissues \cite{Glinos_et_al._2022}, distinguishing host-pathogen interactions 
\cite{Pisu_Huang_Grenier_Russell_2020}, 
and identifying the effect of structural variants on splicing regulators \cite{Pascal_et_al._2023}. \\

Equally important to transcriptomics is abundance quantification. Estimations can be done through an alignment to a 
reference transcriptome sequence; nevertheless, a constant struggle with quantifying eukaryotic samples is to map read 
fragments back to the original transcript as opposed to the original gene. Although the detection of protein family abundances 
can be useful on its own, oftentimes researchers are more so interested in transcript abundances. Since isoforms of the same 
protein will retain common exons, identifying the source of each read is a tedious process that requires 
complex statistical methods.\\

In the past, Salmon was developed to address this particular issue. Given a known transcript and a set of sequenced fragments, 
it quantifies the relative abundance of each transcript in the sample using a dual-phase inference procedure. 
Salmon improves accuracy compared to other models, because it takes into account sample-specific biases to improve accuracy. 
When these biases are not accounted for, calculations like the false discovery rate cannot be controlled for. 
Using multiple inference steps, Salmon improves its abundance estimates using either the VBEM or EM inference algorithms. 
However, both methods have drawbacks. Both algorithms return point estimates of abundances, and we are uncertain about the returned 
estimates without ground truth. Further, the choice of  Bayesian priors used in the VBEM algorithm also has considerations. 
A small prior leads to sparser results than EM. However, a larger prior may result in more estimated non-zero abundance than EM. 
Prior simulated tests \cite{noauthor_salmon_nodate} show that VBEM with a small prior can lead to more accurate estimates. However, 
there has been much research into improving the results of these algorithms such as model averaging \cite{hoeting_bayesian_1999}. 
Given the trade offs of the two algorithms, we sought to know if the averaged prediction for EM and VBEM inference is better than 
one algorithm or another. Further, if this ensemble estimate is better, could it be improved by varying the Bayesian priors.
We have found that the average of these outputs did not result in conclusive results. \\

We saw that using the average is very similar to EM when used with a higher prior and more similar to the VBEM output when used with 
a higher prior. In light of these results we have also included a section on future work. 
